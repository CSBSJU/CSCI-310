\documentclass[10pt,t,svgnames]{beamer}

\usetheme{metropolis} % use metropolis theme

\usepackage{../../solarized}      % use solarized themed listings
\usepackage{../../stack}          % use the tikzstack environment
\usepackage{appendixnumberbeamer} % do not number appendix frames
\usepackage[scale=3]{ccicons}     % creative commons icons

% fix-up the handling of the notes pages
\ifnotes
  \hypersetup{final}
  \usepackage{pgfpages}
  \setbeamertemplate{note page}[plain]
  \setbeameroption{show notes on second screen=right}
  \AtBeginNote{%
    \let\enumerate\itemize%
    \let\endenumerate\enditemize%
  }
\fi

% overrides default description environment
\newlength\wideleftmargin{}
\newlength\tightleftmargin{}
\newlength\diffleftmargin{}
\setlength\wideleftmargin{6em}    % controls location of term (> is more left)
\setlength\tightleftmargin{1.5em} % controls location of description (same)
\setlength\diffleftmargin{\dimexpr\wideleftmargin-\tightleftmargin}
\makeatletter
\providecommand{\nextline}{
  \setlength\labelwidth{\tightleftmargin}
  \setlength\leftmargin{\tightleftmargin}
  \advance\linewidth\diffleftmargin{}
  \advance\@totalleftmargin-\diffleftmargin{}
  \parshape\@ne\@totalleftmargin\linewidth{}
  \setlength\itemsep{1.5ex}
}
\makeatother
\let\origdescription\description
\let\endorigdescription\enddescription
\renewenvironment{description}{\origdescription\nextline}{\endorigdescription}

%-------------------------------------------------------------------------------

\usepackage{tabularx} % tables
\usepackage{booktabs} % cmidrule

\title{Data representation}
\date{}
\author{College of Saint Benedict \& Saint John's University}
\begin{document}
  \maketitle

  \begin{frame}{decimal refresher}
  %{{{1
    \renewcommand{\arraystretch}{2}
    \begin{center}
      \footnotesize
      \only<+>{58036}
      \begin{tabularx}{\textwidth}{XcXcXcXcX}
        \only<+->{$5$ && $8$ && $0$ && $3$ && $6$\\\hline
                  $50000$ & $+$ & $8000$ & $+$ & $0$ & $+$ & $30$ & $+$ & $6$}
        \only<+->{\\\hline
                  $5\times10000$ & $+$ & $8\times1000$ & $+$ & $0\times100$ & $+$ & $3\times10$ & $+$ & $6\times1$}
        \only<+->{\\\hline
                  $5\times10^4$ & $+$ & $8\times10^3$ & $+$ & $0\times10^2$ & $+$ & $3\times10^1$ & $+$ & $6\times10^0$}
      \end{tabularx}
    \end{center}
  %}}}1
  \end{frame}

  \begin{frame}{binary refresher}
  %{{{1
    \renewcommand{\arraystretch}{2}
    \begin{center}
      \footnotesize
      \only<+>{10110}
      \begin{tabularx}{\textwidth}{XcXcXcXcXcXcX}
        \only<+->{$1$ && $0$ && $1$ && $1$ && $0$\\\hline
                  $1\times2^4$ & $+$ & $0\times2^3$ & $+$ & $1\times2^2$ & $+$ & $1\times2^1$ & $+$ & $0\times2^0$}
        \only<+->{\\\hline
                  $1\times16$ & $+$ & $0\times8$ & $+$ & $1\times4$ & $+$ & $1\times2$ & $+$ & $0\times1$}
        \only<+->{\\\hline
                  $16$ & $+$ & $0$ & $+$ & $4$ & $+$ & $2$ & $+$ & $0$}
      \end{tabularx}
    \end{center}

    \note[item]<.->{this is the representation for unsigned binary integers}
    \note[item]<.->{so how to represent signed integers?
      \begin{itemize}
        \item why not use the leftmost bit to store the sign?
          \begin{itemize}
            \item what is the range of values if we choose this? --- 0 is
              represented twice, so our range has one less value --- not the end
              of the world
            \item what happens if we add to -5 to +5? --- the result is -10?
          \end{itemize}
      \end{itemize}}
  %}}}1
  \end{frame}

  \begin{frame}{unsigned addition}
  %{{{1
    \renewcommand{\arraystretch}{2}
    \begin{center}
      \begin{tabular}{rrrrr}
        $0$ & $+$ & $0$ & $=$ & $0$\\
        $0$ & $+$ & $1$ & $=$ & $1$\\
        $1$ & $+$ & $0$ & $=$ & $1$\\
        $1$ & $+$ & $1$ & $=$ & $10$\\
      \end{tabular}

      \hrule

      \begin{tabular}{rrrrrrrrl}
            & $0$ & $0$ && $0$ & $1$ & $0$ & $1$ & $=5$\\
        ADD & $0$ & $0$ && $0$ & $1$ & $0$ & $1$ & $=5$\\\cmidrule{1-8}
        \only<2>{$\mbox{C}=0$ & $0$ & $0$ && $1$ & $0$ & $1$ & $0$ & $=10$}
      \end{tabular}
    \end{center}

    \note[item]<2->{the hardware has a special bit known as the \emph{carry}
      bit, denoted by C, which stores a 1 if the result of the addition was a
      carry, and 0 otherwise.}
  %}}}1
  \end{frame}

  \begin{frame}{signed addition}
  %{{{1
    \renewcommand{\arraystretch}{2}
    \begin{center}
      \begin{tabular}{rrrrrrrrl}
            & $0$ & $0$ && $0$ & $1$ & $0$ & $1$ & $=+5$\\
        ADD & $1$ & $0$ && $0$ & $1$ & $0$ & $1$ & $=-5$\\\cmidrule{1-8}
        \only<2>{$\mbox{C}=0$ & $1$ & $0$ && $1$ & $0$ & $1$ & $0$ & $=-10$}
      \end{tabular}
    \end{center}

    \note[item]<2->{what is the problem
      \begin{itemize}
        \item in this case, we had two additional symbols, $+$ and $-$, and we
          were making some assumptions about their behavior.
        \item For example, we know that $5 \mbox{ADD} 5=10$, but what does $+
          \mbox{ADD} -$ equal? --- we have no rule for that in our definition of
          decimal.
        \item we are trying to apply the addition algorithm, when we should be
          applying a different algorithm, called \emph{subtraction}
        \item can we choose a different representation that we can directly use
          the addition algorithm with?
      \end{itemize}
    }
  %}}}1
  \end{frame}

  \begin{frame}{one's complement}
  %{{{1
    \renewcommand{\arraystretch}{2}
    \begin{center}
      \begin{tabular}{rrrrrrrr}
              NOT & $0$ & $0$ && $0$ & $1$ & $0$ & $1$\\\hline
        \only<2->{& $1$ & $1$ && $1$ & $0$ & $1$ & $0$}
      \end{tabular}

      \only<3->{\hrule}

      \begin{tabular}{rrrrrrrr}
        \only<3->{    & $0$ & $0$ && $0$ & $1$ & $0$ & $1$\\}
        \only<3->{ADD & $1$ & $1$ && $1$ & $0$ & $1$ & $0$\\\hline}
        \only<4->{$\mbox{C}=0$ & $1$ & $1$ && $1$ & $1$ & $1$ & $1$\\}
        \only<5->{ADD & $0$ & $0$ && $0$ & $0$ & $0$ & $1$\\\hline}
        \only<6->{$\mbox{C}=1$ & $0$ & $0$ && $0$ & $0$ & $0$ & $0$}
      \end{tabular}
    \end{center}

    \note[item]<2->{one's complement is known as logical not}
    \note[item]<4->{adding the one's complement will always result in all 1s}
    \note[item]<6->{so two's complement is $\mbox{NOT}+1$}
  %}}}1
  \end{frame}

  %\begin{frame}{overflow bit}
  %%{{{1
  %  \renewcommand{\arraystretch}{2}
  %  \begin{center}
  %    \begin{tabular}{rrrrrrrr}
  %            NOT & $0$ & $0$ && $0$ & $1$ & $0$ & $1$\\\hline
  %      \only<2->{& $1$ & $1$ && $1$ & $0$ & $1$ & $0$}
  %    \end{tabular}

  %    \only<3->{\hrule}

  %    \begin{tabular}{rrrrrrrr}
  %      \only<3->{    & $0$ & $0$ && $0$ & $1$ & $0$ & $1$\\}
  %      \only<3->{ADD & $1$ & $1$ && $1$ & $0$ & $1$ & $0$\\\hline}
  %      \only<4->{$\mbox{C}=0$ & $1$ & $1$ && $1$ & $1$ & $1$ & $1$\\}
  %      \only<5->{ADD & $0$ & $0$ && $0$ & $0$ & $0$ & $1$\\\hline}
  %      \only<6->{$\mbox{C}=1$ & $0$ & $0$ && $0$ & $0$ & $0$ & $0$}
  %    \end{tabular}
  %  \end{center}

  %  \note[item]<2->{one's complement is known as logical not}
  %  \note[item]<4->{adding the one's complement will always result in all 1s}
  %  \note[item]<6->{so two's complement is $\mbox{NOT}+1$}
  %%}}}1
  %\end{frame}

  %\begin{frame}{unsigned addition}
  %%{{{1
  %  \renewcommand{\arraystretch}{2}
  %  \begin{center}
  %    \begin{tabular}{rrrrrrrrl}
  %      & \only<6->{$0$} & \only<5->{$0$} && \only<4->{$1$} & \only<3->{$0$} & \only<2->{$1$} & &      \\
  %      \only<1->{& $0$ & $0$ && $0$ & $1$ & $0$ & $1$ & $=+5$\\
  %            ADD & $1$ & $0$ && $0$ & $1$ & $0$ & $1$ & $=-5$\\\cmidrule{1-8}}
  %      \only<7->{$\mbox{C}=0$} & \only<7->{$1$} & \only<6->{$0$} && \only<5->{$1$} &
  %      \only<4->{$0$} & \only<3->{$1$} & \only<2->{$0$} & \only<7->{$=-10$} \\
  %    \end{tabular}
  %  \end{center}

  %  \note[item]<7->{But this is not how we would add the numbers $+5$ and $-5$
  %    anyway, we would use a different algorithm called \emph{subtraction}}


  %  \note[item]<.->{this is the representation for unsigned binary integers}
  %  \note[item]<.->{so how to represent signed integers?
  %    \begin{itemize}
  %      \item why not use the leftmost bit to store the sign?
  %        \begin{itemize}
  %          \item what is the range of values if we choose this? --- 0 is
  %            represented twice, so our range has one less value --- not the end
  %            of the world
  %          \item what happens if we add to -5 to +5? --- the result is -10?
  %        \end{itemize}
  %    \end{itemize}}
  %%}}}1
  %\end{frame}

  %\begin{frame}{signed integers}
  %%{{{1
  %  \renewcommand{\arraystretch}{2}
  %  \begin{center}
  %    \footnotesize
  %    \only<+>{10110}
  %    \begin{tabularx}{\textwidth}{XcXcXcXcXcXcX}
  %      \only<+->{$1$ && $0$ && $1$ && $1$ && $0$\\\hline
  %                $1\times2^4$ & $+$ & $0\times2^3$ & $+$ & $1\times2^2$ & $+$ & $1\times2^1$ & $+$ & $0\times2^0$}
  %      \only<+->{\\\hline
  %                $1\times16$ & $+$ & $0\times8$ & $+$ & $1\times4$ & $+$ & $1\times2$ & $+$ & $0\times1$}
  %      \only<+->{\\\hline
  %                $16$ & $+$ & $0$ & $+$ & $4$ & $+$ & $2$ & $+$ & $0$}
  %    \end{tabularx}
  %  \end{center}

  %  \note[item]<.->{test}
  %%}}}1
  %\end{frame}

  \appendix

  \begin{frame}[c]
  %{{{1
    \begin{center}\ccbysa\end{center}

    except where otherwise noted, this worked is licensed under
    \href{http://creativecommons.org/licenses/by-sa/4.0/}{creative commons
    attribution-sharealike 4.0 international license}
  %}}}1
  \end{frame}
\end{document}
