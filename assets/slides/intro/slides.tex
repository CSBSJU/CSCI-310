\ifnotes
\else
  \PassOptionsToClass{handout}{beamer}
\fi

\documentclass[10pt,t,svgnames]{beamer}

\usetheme{metropolis} % use metropolis theme

\usepackage{../solarized}         % use solarized themed listings
\usepackage{../stack}             % use the tikzstack environment
\usepackage{appendixnumberbeamer} % do not number appendix frames
\usepackage[scale=3]{ccicons}     % creative commons icons

% fix-up the handling of the notes pages
\ifnotes
  \hypersetup{final}
  \usepackage{pgfpages}
  \setbeamertemplate{note page}[plain]
  \setbeameroption{show notes on second screen=right}
  \AtBeginNote{%
    \let\enumerate\itemize%
    \let\endenumerate\enditemize%
  }
\fi

% overrides default description environment
\newlength\wideleftmargin{}
\newlength\tightleftmargin{}
\newlength\diffleftmargin{}
\setlength\wideleftmargin{6em}    % controls location of term (> is more left)
\setlength\tightleftmargin{1.5em} % controls location of description (same)
\setlength\diffleftmargin{\dimexpr\wideleftmargin-\tightleftmargin}
\makeatletter
\providecommand{\nextline}{
  \setlength\labelwidth{\tightleftmargin}
  \setlength\leftmargin{\tightleftmargin}
  \advance\linewidth\diffleftmargin{}
  \advance\@totalleftmargin-\diffleftmargin{}
  \parshape\@ne\@totalleftmargin\linewidth{}
  \setlength\itemsep{1.5ex}
}
\makeatother
\let\origdescription\description
\let\endorigdescription\enddescription
\renewenvironment{description}{\origdescription\nextline}{\endorigdescription}

%-------------------------------------------------------------------------------

\usepackage{tabularx}   % tables
\usepackage{subcaption} % subfigures
\usepackage{tikz}       % tikz
\usetikzlibrary{decorations.pathreplacing,positioning}

\tikzset{
  lstanchor/.style={
    inner sep=0pt,
    text width=0pt,
    align=center,
    minimum height=1.25ex,
    node distance=0pt,
  }
}

\title{Computer systems}
\date{}
\author{}
\institute{College of Saint Benedict \& Saint John's University}
\begin{document}
  \maketitle

  \begin{frame}[c]{great insights of computer science\footnotemark}
  %{{{1
    \begin{description}
      \item[Bacon, Leibniz, Boole, Turing, Shannon, \& Morse] \hfill \\
        There are only \textbf{two nouns} that a computer has to deal with in
        order to represent ``anything'': 0, 1.
    \end{description}

    \note[item]{``anything'': there are some things computers cannot do --- like
      determine if a program will ever finish.}

    \footnotetext[1]{\href{https://en.wikipedia.org/wiki/Computer\_science\#The\_great\_insights\_of\_computer\_science}{The great insights of computer science}~/~\href{http://creativecommons.org/licenses/by-sa/3.0}{CC~BY-SA~3.0}}
  %}}}1
  \end{frame}

  \begin{frame}[c]{great insights of computer science, cont'd}
  %{{{1
    \begin{description}
      \item[Turing] \hfill \\
        There are only \textbf{five verbs} that a computer has to perform in
        order to do ``anything'':
        \begin{enumerate}
          \item move left one location;
          \item move right one location;
          \item read symbol at current location;
          \item print 0 at current location;
          \item print 1 at current location.
        \end{enumerate}
    \end{description}
  %}}}1
  \end{frame}

  \begin{frame}[c]{great insights of computer science, cont'd}
  %{{{1
    \begin{description}
      \item[Boehm and Jacopini] \hfill \\
        There are only \textbf{three grammar rules} needed to combine these
        verbs (into more complex ones) that are needed in order for a computer
        to do "anything":
        \begin{enumerate}
          \item \emph{sequence}: first do this, then do that;
          \item \emph{selection}: IF such-and-such is the case, THEN do this,
            ELSE do that;
          \item \emph{repetition}: WHILE such-and-such is the case DO this.
        \end{enumerate}
    \end{description}
  %}}}1
  \end{frame}

  \begin{frame}[fragile]{great insights of computer science, cont'd}
  %{{{1
    \begin{enumerate}
      \item two nouns
      \item five verbs
      \item three grammar rules
    \end{enumerate}
    \vspace{\baselineskip}

    \begin{tabular}{l|l}
      \hline
      \texttt{\textless}    & move left one location\\
      \texttt{\textgreater} & move right one location\\
      \texttt{0}            & print \texttt{0} at current location\\
      \texttt{1}            & print \texttt{1} at current location\\
      \texttt{[}            & if current location is \texttt{0}, then go to
                              instruction after matching \texttt{]}\\
      \texttt{]}            & go to matching \texttt{[} instruction\\
      \hline
    \end{tabular}

    \begin{termblock}
    1>1>0>1>0<<<<[0>]1
    \end{termblock}

    \note[item]{\emph{sequence}: start at left-most instruction and progress a
      single instruction to the right}
    \note[item]{\emph{selection} and \emph{repetition:}
      \texttt{[}\dots\texttt{]} provide both --- repetition is just fancy
      selection}
  %}}}1
  \end{frame}

  % TODO
  % * control abstraction
  %   * addition (small), multiplication (medium), power (large)
  % * data abstraction
  %   * numerical data types
  %   * character types
  %   * boolean type (nice abstraction, because it is really just a number)
  % history of computing abstraction
  %   * key people involved at each level

  \begin{frame}{comparison}
  %{{{1
    \begin{center}
      \begin{tabularx}{.8\textwidth}{XX}
        \hline
        \textbf{Java} & \textbf{C}\\
        \hline
        object-oriented & procedural\\
        interpreted & compiled\\
        \texttt{String} & \texttt{char} array\\
        condition (\texttt{boolean}) & condition (\texttt{int})\\
        garbage-collected & no memory management\\
        references & pointers\\
        exceptions & error codes\\
        \hline
      \end{tabularx}
    \end{center}

    \note{
      \begin{itemize}
        \item in Java, everything is a method that is called on an object
        \item in C, everything is a function
      \end{itemize}
      \begin{itemize}
        \item in Java, source code is compiled to byte code, which is then
          interpreted by Java VM
        \item in C, source code is compiled into binary machine code
      \end{itemize}
      \begin{itemize}
        \item in Java, String is a class
        \item in C, a string is just an array of \texttt{char} values which ends
          with the \texttt{char~'\textbackslash0'}
      \end{itemize}
      \begin{itemize}
        \item in Java, the Java VM takes care of deallocating memory used
        \item in C, any memory you allocate, you must also deallocate
      \end{itemize}
    }
  %}}}1
  \end{frame}

  \begin{frame}[fragile]{hello, world}
  %{{{1
    \begin{codeblock}
    /* file: helloworld.c */

    ###include## $$<stdio.h>$$

    int main() {
      printf([["hello, world]]%%\n%%[["]]);
      return $$0$$;
    }
    \end{codeblock}

    \begin{termblock}
    $ gcc -o helloworld helloworld.c
    $ ./helloworld
    hello, world
    \end{termblock}

    \note[item]{The tradition of using the phrase "Hello, world!" as a test
      message was influenced by an example program in the seminal book \emph{The
      C Programming Language}}
  %}}}1
  \end{frame}

  \begin{frame}[fragile]{global variables}
  %{{{1
    \begin{scriptsize}
      \begin{codeblock}
      // file: figure2-4.c
      // Stan Warford
      // A nonsense program to illustrate global variables

      ###include## $$<stdio.h>$$

      char ch;
      int j;

      int main() {
        scanf([["]]%%%c%%[[ ]]%%%d%%[["]], &ch, &j);
        j += $$5$$;
        ch++;
        printf([["]]%%%c\n%d\n%%[["]], ch, j);
        return $$0$$;
      }
      \end{codeblock}
    \end{scriptsize}

    \begin{scriptsize}
      \begin{termblock}
      $ gcc -o figure2-4 figure2-4.c
      $ ./figure2-4
      M 419
      N
      424
      \end{termblock}
    \end{scriptsize}

    \note[item]{What would you expect for input '\texttt{Z~-3}'?}
    \note[item]{What would you expect for input '\texttt{9~a}'?}
    \note[item]{What would you expect for input
      '\texttt{\textasciitilde~2147483643}'?}
  %}}}1
  \end{frame}

  \begin{frame}[fragile]{program breakdown}
  %{{{1
    \vspace{1\baselineskip}

    \begin{scriptsize}
      \begin{codeblock}[escapechar=!,xleftmargin=.45\linewidth,firstnumber=5]
      ###include## $$<stdio.h>$$!\tikz[remember picture] \coordinate (f);!

      !\tikz[remember picture] \coordinate (a);!char ch;
      !\tikz[remember picture] \coordinate (b);!int j;

      int main() {!\tikz[remember picture] \coordinate (c);!
        !\tikz[remember picture] \coordinate (i);!scanf([["]]%%%c%%[[ ]]%%%d%%[["]], &ch, &j);!\tikz[remember picture] \coordinate (g);!
        j += $$5$$;
        !\tikz[remember picture] \coordinate (d);!ch++;
        !\tikz[remember picture] \coordinate (h);!printf([["]]%%%c\n%d\n%%[["]], ch, j);!\tikz[remember picture] \coordinate (j);!
        return $$0$$;!\tikz[remember picture] \coordinate (e);!
      }
      \end{codeblock}
        \begin{tikzpicture}[remember picture, overlay,
            every edge/.append style={
              ->,
              thick,
              >=stealth,
              DimGray,
              dashed,
              thick,
            },
            every node/.append style={
              draw,
              align=center,
              minimum height=10pt,
              color=DimGray,
              text=fg,
              thin,
              rounded corners,
              text width=.25\linewidth,
            },
            group/.style = {
              decoration={
                brace,
                mirror,
                raise=5pt,
              },
              thick,
              color=DimGray,
              decorate,
            },
          ]
          \only<1>{
            \node[] at ([shift=({19.5em,.625ex})]c) (C) {C programs ALWAYS start execution with the \texttt{main} function};
            \node[] at ([shift=({20em,.625ex})]e) (E) {returning from \texttt{main} ends the program};
            \draw (C.west) edge ([shift=({.4em,.625ex})]c);
            \draw (E.west) edge ([shift=({.4em,.625ex})]e);
          }
          \only<2>{
            \draw[group] ([shift=({.4em,1.65ex})]a) -- node[left=4.5em] (AB) {global variables are declared here --- outside of any function} ([shift=({.4em,-.4ex})]b);
            \node[] at ([shift=({-11em,.625ex})]d) (D) {characters in C are treated internally like signed integers};
            \draw (AB.east) edge ([xshift=3em]AB.east);
            \draw (D.east) edge ([shift=({-.4em,.625ex})]d);
          }
          \only<3>{
            \node[] at ([shift=({16em,.625ex})]f) (F) {correct headers must be included to access library functions};
            \node[] at ([shift=({-11em,.625ex})]h) (H) {print data to \texttt{stdout} (the terminal)};
            \node[] at ([shift=({-11em,.625ex})]i) (I) {read data from \texttt{stdin} (the terminal)};
            \node[] at ([shift=({11em,-3.625ex})]g) (GJ) {\texttt{scanf} and \texttt{printf} are both library functions declared in \texttt{stdio.h}};
            \draw (F.west) edge ([shift=({.4em,.625ex})]f);
            \draw (GJ.west) edge ([shift=({.4em,.625ex})]g);
            \draw (GJ.west) edge ([shift=({.4em,.625ex})]j);
            \draw (H.east) edge ([shift=({-.4em,.625ex})]h);
            \draw (I.east) edge ([shift=({-.4em,.625ex})]i);
          }
          \only<4>{
            \node[] at ([shift=({11em,.625ex})]g) (G) {\texttt{\&} is the address of operator --- \texttt{scanf} expects the address of the variables where the data will be stored};
            \draw (G.west) edge ([shift=({.4em,.625ex})]g);
          }
        \end{tikzpicture}
      \end{scriptsize}

      \note[item]<3>{C has no ``built-in'' functions; however, it does have a
        standard library that includes many useful utility functions.}
  %}}}1
  \end{frame}

  \begin{frame}{memory model --- part i}
  %{{{1
    \begin{description}
      \item[global variables] \hfill \\ declared outside of any function and
        remain in place throughout the execution of the entire program. they are
        stored at a fixed location in memory.
      \item[local variables] \hfill \\ declared within a function and come into
        existence when the function is called and cease to exist when the
        function terminates. they are stored on the run-time stack.
    \end{description}

    \vspace{1\baselineskip}

    \begin{figure}
      \centering
      \begin{subfigure}[b]{.30\textwidth}
        \centering
        \begin{tikzstack}[bg]
          \begin{stack}{west}
            % global variables
            \variables{j/424,ch/N}
          \end{stack}
        \end{tikzstack}
        \caption{Fixed location.}
      \end{subfigure}
      \quad
      \begin{subfigure}[b]{.30\textwidth}
        \centering
        \begin{tikzstack}[fg]
          \begin{stack}{east}
            % stack variables
            \variables{retVal/0,retAddr/ra0}
            % stack frames
            \frames{0/2}
          \end{stack}
        \end{tikzstack}
        \caption{Run-time stack.}
      \end{subfigure}
    \end{figure}

    \note{
      \begin{itemize}
        \item I will be using graphical notation consistent with that of the
          book.
        \item In this case, (a) and (b) represent the state of relevant memory
          for the previous program just before it terminates, i.e., in the
          process of executing line 15.
        \item How would the previous program behave had it declared \texttt{ch}
          and \texttt{j} as local variables instead of global variables?
        \item What would the memory model look like given the above?\\[1\baselineskip]

        \begin{figure}
          \begin{tikzstack}[fg]
            \begin{stack}{east}
              % stack variables
              \variables{retVal/0,retAddr/ra0,j/424,ch/N}
              % stack frames
              \frames{0/4}
            \end{stack}
          \end{tikzstack}
          \caption{Run-time stack.}
        \end{figure}
      \end{itemize}
    }
  %}}}1
  \end{frame}

  \begin{frame}{FIXME --- tikzstack testing}
  %{{{1
    \begin{figure}
      \centering
      \begin{subfigure}[b]{.50\textwidth}
        \centering
        \begin{tikzstack}[bg]
          \begin{stack}{west}
            % global variables
            \variables{j/424,ch/N,ptr2/}
            % valid pointers
            %\pointers{3/1,3/2}
            % null pointers
            \nulls{3}
          \end{stack}
          \begin{stack}[xshift=7em]{east}
            % global variables
            \variables{a/,b/1,c/}
            % valid pointers
            \pointers{1/1-2,3/1-2}
          \end{stack}
        \end{tikzstack}
        \caption{Fixed location.}
      \end{subfigure}
      \quad
      \begin{subfigure}[b]{.30\textwidth}
        \centering
        \begin{tikzstack}[fg]
          \begin{stack}{east}
            % stack variables
            \variables{retVal/,retAddr/ra0,n/3,retAddr/ra1,k/}
            % stack frames
            \frames{0/2,2/3}
          \end{stack}
        \end{tikzstack}
        \caption{Run-time stack.}
      \end{subfigure}
    \end{figure}
  %}}}1
  \end{frame}

  \begin{frame}[fragile]{conditions}
  %{{{1
    \begin{itemize}
      \item under what conditions will each of the following be execute?
    \end{itemize}
    \begin{codeblock}
    if (x) {
      /* ??? */
    }
    if (x-y) {
      /* ??? */
    }
    if (x=y) {
      /* ??? */
    }
    \end{codeblock}

    \note{
      \begin{itemize}
        \item x != 0
        \item x != y
        \item y != 0
      \end{itemize}
    }
  %}}}1
  \end{frame}

%  \begin{frame}{add evens}
%    \begin{itemize}
%      \item create program called \texttt{add\_even.c} that adds all the even
%        numbers between 1 and 100 and prints the sum
%    \end{itemize}
%  \end{frame}
%
%  \begin{frame}[fragile]{cli}
%    \begin{codeblock}
%    ###include## $$<stdio.h>$$
%
%    int main(int argc, char * argv[]) {
%      printf([["(]]%%%d%%[[) ]]%%%s%%[[:]]%%%s\n%%[["]], argc, argv[$$0$$],
%        argv[$$1$$]);
%      return $$0$$;
%    }
%    \end{codeblock}
%
%    \begin{itemize}
%      \item modify \texttt{add\_even.c} to get maximum value from the
%        command-line instead of hard-coded as 100
%    \end{itemize}
%  \end{frame}
%
%  \begin{frame}[fragile]{printf / scanf}
%    \begin{itemize}
%      \item \texttt{printf()} interprets variables and prints character
%        representations to standard out (usually the terminal)
%      \item \texttt{scanf()} scans characters from standard in (usually the
%        terminal) and interprets them for storage in variables
%    \end{itemize}
%
%    \begin{codeblock}
%    ###include## $$<stdio.h>$$
%
%    int main() {
%      int i;
%      scanf([["]]%%%d%%[["]], &i);
%      return $$0$$;
%    }
%    \end{codeblock}
%
%    \note{
%      \begin{itemize}
%        \item scanf requires you to pass the address of the variable, so that
%          its value can be changed
%      \end{itemize}
%    }
%  \end{frame}
%
%  \begin{frame}[fragile]{echo}
%    \begin{itemize}
%      \item modify \texttt{helloworld.c} to ask user for an input and then print
%        it back
%      \item change its name to \texttt{echo.c}
%    \end{itemize}
%
%    \begin{termblock}
%    $ ./echo
%    Enter a string to echo: hello, world
%    hello,
%    \end{termblock}
%
%    \note{
%      \begin{itemize}
%        \item why did it print \texttt{hello,} instead of \texttt{hello, world}?
%      \end{itemize}
%    }
%  \end{frame}
%
%  \begin{frame}[fragile]{pointers}
%    \begin{itemize}
%      \item a pointer is a variable whose value is a memory address
%    \end{itemize}
%    \begin{codeblock}
%    int   i  = $$0x1A$$;
%    int * ip = &i;
%    \end{codeblock}
%    \begin{itemize}
%      \item \texttt{\&i} evaluates to the address where the variable \texttt{i}
%        is stored in memory
%      \item \texttt{i} is an \texttt{int}, so \texttt{ip} is a \emph{pointer} to
%        an \texttt{int}
%    \end{itemize}
%
%    \hspace{8pt}%
%    \begin{tikzpicture}[font=\ttfamily,noname/.style={text height=1.5ex, text depth=.25ex, text centered, minimum height=3em}]
%      \node[noname] at (-1.3,.3) {\textbf{0x000012A0}};
%      \foreach \x in {0,.6,...,2.4}
%        \draw (\x,0) rectangle ++(.6,.6);
%      \draw[bg, ultra thick] (-.1,.6) -- ++(3.2,0);
%      \foreach [count=\i,evaluate=\i as \x using .3+(\i-1)*.6] \c in {00,00,00,1A}
%        \node[noname] at (\x,.3) {\c};
%      \draw[decoration={brace,mirror,raise=5pt},decorate] (2.4,0) -- node[right=6pt] {i} ++(0,.6);
%
%      \node[noname] at (-1.3,-.7) {\textbf{0x????????}};
%      \foreach \x in {0,.6,...,2.4}
%        \draw (\x,-1) rectangle ++(.6,.6);
%      \draw[bg, ultra thick] (-.1,-.4) -- ++(3.2,0);
%      \foreach [count=\i,evaluate=\i as \x using .3+(\i-1)*.6] \c in {00,00,12,A0}
%        \node[noname] at (\x,-.7) {\c};
%      \draw[decoration={brace,mirror,raise=5pt},decorate] (2.4,-1) -- node[right=6pt] {ip} ++(0,.6);
%    \end{tikzpicture}
%  \end{frame}
%
%  \begin{frame}[fragile]{pointers cont.}
%    \begin{codeblock}
%    printf([["0x]]%%%X\n%%[["]], i);    /* 0x1A */
%    printf([["0x]]%%%#X\n%%[["]], &i);  /* 0x12A0 */
%    printf([["0x]]%%%#X\n%%[["]], ip);  /* 0x12A0 */
%    printf([["0x]]%%%#X\n%%[["]], &ip); /* 0x???????? */
%    \end{codeblock}
%
%    \note{
%      \begin{itemize}
%        \item so how can we use the pointer, \texttt{ip}, to access the value of
%          \texttt{i}?
%      \end{itemize}
%    }
%  \end{frame}
%
%  \begin{frame}[fragile]{pointer dereference}
%    \begin{itemize}
%      \item \texttt{*ptr} will
%        \begin{enumerate}
%          \item treat the value of \texttt{ptr} as a memory address
%          \item get the bytes of data located at that memory address
%          \item interpret those bytes according to the type of pointer that
%            \texttt{ptr} is
%        \end{enumerate}
%    \end{itemize}
%    \begin{codeblock}
%    printf([["0x]]%%%X\n%%[["]], *ip);   /* 0x1A */
%    \end{codeblock}
%
%    \begin{onlyenv}<2->
%      \begin{itemize}
%        \item \texttt{ip[X] = *(ip + X)}
%      \end{itemize}
%      \begin{codeblock}
%    printf([["0x]]%%%X\n%%[["]], ip[0]); /* 0x1A */
%      \end{codeblock}
%    \end{onlyenv}
%
%    \note{
%      \begin{itemize}
%        \item the C compiler is ''smart enough'' to ''know'' that \texttt{+ X}
%          really means add \texttt{X * sizeof(*ip)} to \texttt{ip}
%      \end{itemize}
%    }
%  \end{frame}
%
%  \begin{frame}[fragile]{pointers cont.}
%    \begin{codeblock}
%    printf([["0x]]%%%X\n%%[["]], i);       /* 0x1A */
%    printf([["0x]]%%%X\n%%[["]], *ip);     /* 0x1A */
%    printf([["0x]]%%%X\n%%[["]], ip[$$0$$]);   /* 0x1A */
%    printf([["0x]]%%%X\n%%[["]], *(ip+$$0$$)); /* 0x1A */
%    printf([["0x]]%%%#X\n%%[["]], &i);     /* 0x12A0 */
%    printf([["0x]]%%%#X\n%%[["]], ip);     /* 0x12A0 */
%    printf([["0x]]%%%#X\n%%[["]], &ip);    /* 0x???????? */
%    \end{codeblock}
%  \end{frame}
%
%
%  \begin{frame}[fragile]{pointers cont.}
%    \begin{codeblock}
%    char * cp = "hello, world";
%    \end{codeblock}
%    \begin{itemize}
%      \item \texttt{cp} is a \emph{pointer} to a \texttt{char}
%    \end{itemize}
%
%    \vspace{-14pt}%
%    \hspace{8pt}%
%    \begin{tikzpicture}[font=\ttfamily,noname/.style={text height=1.5ex, text depth=.25ex, text centered, minimum height=3em}]
%      \node[noname] at (-1.3,.3) {\textbf{0x00004C80}};
%      \foreach \x in {0,.6,...,7.8}
%        \draw (\x,0) rectangle ++(.6,.6);
%      \draw[bg, ultra thick] (-.1,.6) -- ++(8.4,0);
%      \foreach [count=\i,evaluate=\i as \x using .3+(\i-1)*.6] \c in {h,e,l,l,o,{,}, ,w,o,r,l,d,\textbackslash0}
%        \node[noname] at (\x,.3) {\c};
%
%      \node[noname] at (-1.3,-.7) {\textbf{0x????????}};
%      \foreach \x in {0,.6,...,2.4}
%        \draw (\x,-1) rectangle ++(.6,.6);
%      \draw[bg, ultra thick] (-.1,-.4) -- ++(3.2,0);
%      \foreach [count=\i,evaluate=\i as \x using .3+(\i-1)*.6] \c in {00,00,4C,80}
%        \node[noname] at (\x,-.7) {\c};
%    \end{tikzpicture}
%
%    \vspace{-6pt}%
%    \begin{codeblock}
%    printf([["]]%%%c\n%%[["]], *cp);     /* h */
%    printf([["]]%%%c\n%%[["]], cp[$$0$$]);   /* h */
%    printf([["]]%%%c\n%%[["]], cp[$$4$$]);   /* o */
%    printf([["]]%%%c\n%%[["]], *(cp+$$4$$)); /* o */
%    printf([["]]%%%s\n%%[["]], cp);      /* hello, world */
%    printf([["]]%%%s\n%%[["]], cp+7);    /* world */
%    printf([["0x]]%%%#X\n%%[["]], cp);   /* 0x4C80 */
%    printf([["0x]]%%%#X\n%%[["]], &cp);  /* 0x???????? */
%    \end{codeblock}
%
%    \note{
%      \begin{itemize}
%        \item why not say \texttt{cp} is a \emph{pointer} to a \texttt{char}
%          array?
%      \end{itemize}
%    }
%  \end{frame}
%
%  \begin{frame}[fragile]{practice}
%    \begin{codeblock}
%    ###include## $$<stdio.h>$$
%
%    void swap(int n1, int n2) {
%      int tmp = n1;
%      n1 = n2;
%      n2 = tmp;
%    }
%
%    int main() {
%      int v1 = $$11$$, v2 = $$77$$;
%      printf([["BEFORE  v1=]]%%%d%%[[, v2=]]%%%d\n%%[["]], v1, v2);
%      swap(v1, v2);
%      printf([["AFTER  v1=]]%%%d%%[[, v2=]]%%%d\n%%[["]], v1, v2);
%      return $$0$$;
%    }
%    \end{codeblock}
%
%    \note{
%      \begin{itemize}
%        \item what's wrong with this program?
%        \item fix the program so that it correctly swaps the two variables'
%          values
%      \end{itemize}
%    }
%  \end{frame}
%
%  \begin{frame}[fragile]{heap memory}
%    \begin{itemize}
%      \item designate a block of memory to store value(s) of a particular
%        data type
%    \end{itemize}
%    \begin{codeblock}
%    int * ip = malloc($$100$$*sizeof(int));
%    \end{codeblock}
%
%    \hspace{8pt}%
%    \begin{tikzpicture}[font=\ttfamily,noname/.style={text height=1.5ex, text depth=.25ex, text centered, minimum height=3em}]
%      \node[noname] at (-1.3,.3) {\textbf{0x000063DA}};
%      \foreach \x in {0,.6,...,7.2}
%        \draw (\x,0) rectangle ++(.6,.6);
%      \draw[bg, ultra thick] (-.1,.6) -- ++(8,0);
%      \foreach [count=\i,evaluate=\i as \x using .3+(\i-1)*.6] \c in {r,@,!,X,t,v,9,1,S,?,{)},.}
%        \node[noname] at (\x,.3) {\c};
%      \node at (7.6,.3) {$\boldsymbol{\cdots}$};
%
%      \node[noname] at (-1.3,-.7) {\textbf{0x????????}};
%      \foreach \x in {0,.6,...,2.4}
%        \draw (\x,-1) rectangle ++(.6,.6);
%      \draw[bg, ultra thick] (-.1,-.4) -- ++(3.2,0);
%      \foreach [count=\i,evaluate=\i as \x using .3+(\i-1)*.6] \c in {00,00,63,DA}
%        \node[noname] at (\x,-.7) {\c};
%    \end{tikzpicture}
%
%    \begin{onlyenv}<2->
%      \begin{itemize}
%        \item release a block of memory back to system to be used elsewhere
%      \end{itemize}
%      \begin{codeblock}
%    free(ip);
%      \end{codeblock}
%    \end{onlyenv}
%
%    \note{
%      \begin{itemize}
%        \item allocates enough consecutive memory for 100 \texttt{int} values
%      \end{itemize}
%    }
%  \end{frame}
%
%  \begin{frame}[fragile]{heap memory cont.}
%    \begin{codeblock}
%    ip[$$0$$] = $$0x7$$; /* *ip = 0x7; */
%    \end{codeblock}
%
%    \hspace{8pt}%
%    \begin{tikzpicture}[font=\ttfamily,noname/.style={text height=1.5ex, text depth=.25ex, text centered, minimum height=3em}]
%      \node[noname] at (-1.3,.3) {\textbf{0x000063DA}};
%      \foreach \x in {0,.6,...,7.2}
%        \draw (\x,0) rectangle ++(.6,.6);
%      \draw[bg, ultra thick] (-.1,.6) -- ++(8,0);
%      \foreach [count=\i,evaluate=\i as \x using .3+(\i-1)*.6] \c in {00,00,00,07,t,v,9,1,S,?,{)},.}
%        \node[noname] at (\x,.3) {\c};
%      \node at (7.6,.3) {$\boldsymbol{\cdots}$};
%
%      \node[noname] at (-1.3,-.7) {\textbf{0x????????}};
%      \foreach \x in {0,.6,...,2.4}
%        \draw (\x,-1) rectangle ++(.6,.6);
%      \draw[bg, ultra thick] (-.1,-.4) -- ++(3.2,0);
%      \foreach [count=\i,evaluate=\i as \x using .3+(\i-1)*.6] \c in {00,00,63,DA}
%        \node[noname] at (\x,-.7) {\c};
%    \end{tikzpicture}
%
%    \begin{onlyenv}<2->
%      \begin{codeblock}
%    ip[$$1$$] = $$0xA$$; /* *(ip + 1) = 0xA; */
%      \end{codeblock}
%
%      \hspace{8pt}%
%      \begin{tikzpicture}[font=\ttfamily,noname/.style={text height=1.5ex, text depth=.25ex, text centered, minimum height=3em}]
%        \node[noname] at (-1.3,.3) {\textbf{0x000063DA}};
%        \foreach \x in {0,.6,...,7.2}
%          \draw (\x,0) rectangle ++(.6,.6);
%        \draw[bg, ultra thick] (-.1,.6) -- ++(8,0);
%        \foreach [count=\i,evaluate=\i as \x using .3+(\i-1)*.6] \c in {00,00,00,07,00,00,00,0A,S,?,{)},.}
%          \node[noname] at (\x,.3) {\c};
%        \node at (7.6,.3) {$\boldsymbol{\cdots}$};
%      \end{tikzpicture}
%    \end{onlyenv}
%  \end{frame}
%
%  \begin{frame}[fragile]{file i/o}
%    \begin{codeblock}
%    ###include## $$<stdio.h>$$
%
%    int main(int argc, char * argv[]) {
%      int m, n;
%      FILE * fp;
%
%      if (!(fp = fopen("example.txt", "r")))
%        return $$-1$$;
%      if ($$2$$ != fscanf(fp, [["]]%%%d %d%%[["]], &m, &n))
%        return $$-1$$;
%      if (!fclose(fp))
%        return $$-1$$;
%
%      return $$0$$;
%    }
%    \end{codeblock}
%  \end{frame}
%
%  \begin{frame}{final thoughts}
%    \begin{itemize}
%      \item matrices
%    \end{itemize}
%  \end{frame}

  %\begin{frame}[fragile]{practice}
  %  \begin{itemize}
  %    \item download \href{}{this} code
  %    \item implement the \texttt{swap} method
  %  \end{itemize}
  %\end{frame}

  \appendix

  \begin{frame}[c]
    \begin{center}\ccbysa\end{center}

    except where otherwise noted, this worked is licensed under
    \href{http://creativecommons.org/licenses/by-sa/4.0/}{creative commons
    attribution-sharealike 4.0 international license}
  \end{frame}
\end{document}
