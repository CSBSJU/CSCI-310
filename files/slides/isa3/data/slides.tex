\documentclass[10pt,t,svgnames]{beamer}

\usetheme{metropolis} % use metropolis theme

\usepackage{../../solarized}      % use solarized themed listings
\usepackage{../../stack}          % use the tikzstack environment
\usepackage{appendixnumberbeamer} % do not number appendix frames
\usepackage[scale=3]{ccicons}     % creative commons icons

% fix-up the handling of the notes pages
\ifnotes
  \hypersetup{final}
  \usepackage{pgfpages}
  \setbeamertemplate{note page}[plain]
  \setbeameroption{show notes on second screen=right}
  \AtBeginNote{%
    \let\enumerate\itemize%
    \let\endenumerate\enditemize%
  }
\fi

% overrides default description environment
\newlength\wideleftmargin{}
\newlength\tightleftmargin{}
\newlength\diffleftmargin{}
\setlength\wideleftmargin{6em}    % controls location of term (> is more left)
\setlength\tightleftmargin{1.5em} % controls location of description (same)
\setlength\diffleftmargin{\dimexpr\wideleftmargin-\tightleftmargin}
\makeatletter
\providecommand{\nextline}{
  \setlength\labelwidth{\tightleftmargin}
  \setlength\leftmargin{\tightleftmargin}
  \advance\linewidth\diffleftmargin{}
  \advance\@totalleftmargin-\diffleftmargin{}
  \parshape\@ne\@totalleftmargin\linewidth{}
  \setlength\itemsep{1.5ex}
}
\makeatother
\let\origdescription\description
\let\endorigdescription\enddescription
\renewenvironment{description}{\origdescription\nextline}{\endorigdescription}

%-------------------------------------------------------------------------------

\usepackage{tabularx} % tables
\usepackage{booktabs} % cmidrule
\usepackage{fontspec} % language fonts
\usepackage{luatexja-fontspec}

\title{Data representation}
\date{}
\author{College of Saint Benedict \& Saint John's University}
\begin{document}
  \maketitle

  \begin{frame}{decimal refresher}
  %{{{1
    \renewcommand{\arraystretch}{2}
    \begin{center}
      \footnotesize
      \only<+>{58036}
      \begin{tabularx}{\textwidth}{XcXcXcXcX}
        \only<+->{$5$ && $8$ && $0$ && $3$ && $6$\\\hline
                  $50000$ & $+$ & $8000$ & $+$ & $0$ & $+$ & $30$ & $+$ & $6$}
        \only<+->{\\\hline
                  $5\times10000$ & $+$ & $8\times1000$ & $+$ & $0\times100$ & $+$ & $3\times10$ & $+$ & $6\times1$}
        \only<+->{\\\hline
                  $5\times10^4$ & $+$ & $8\times10^3$ & $+$ & $0\times10^2$ & $+$ & $3\times10^1$ & $+$ & $6\times10^0$}
      \end{tabularx}
    \end{center}
  %}}}1
  \end{frame}

  \begin{frame}{binary refresher}
  %{{{1
    \renewcommand{\arraystretch}{2}
    \begin{center}
      \footnotesize
      \only<+>{10110}
      \begin{tabularx}{\textwidth}{XcXcXcXcXcXcX}
        \only<+->{$1$ && $0$ && $1$ && $1$ && $0$\\\hline
                  $1\times2^4$ & $+$ & $0\times2^3$ & $+$ & $1\times2^2$ & $+$ & $1\times2^1$ & $+$ & $0\times2^0$}
        \only<+->{\\\hline
                  $1\times16$ & $+$ & $0\times8$ & $+$ & $1\times4$ & $+$ & $1\times2$ & $+$ & $0\times1$}
        \only<+->{\\\hline
                  $16$ & $+$ & $0$ & $+$ & $4$ & $+$ & $2$ & $+$ & $0$}
      \end{tabularx}
    \end{center}

    \note[item]<.->{this is the representation for unsigned binary integers}
    \note[item]<.->{so how to represent signed integers?
      \begin{itemize}
        \item why not use the leftmost bit to store the sign?
          \begin{itemize}
            \item what is the range of values if we choose this? --- 0 is
              represented twice, so our range has one less value --- not the end
              of the world
            \item what happens if we add to -5 to +5? --- the result is -10?
          \end{itemize}
      \end{itemize}}
  %}}}1
  \end{frame}

  \begin{frame}{unsigned addition}
  %{{{1
    \renewcommand{\arraystretch}{2}
    \begin{center}
      \begin{tabular}{rrrrr}
        $0$ & $+$ & $0$ & $=$ & $0$\\
        $0$ & $+$ & $1$ & $=$ & $1$\\
        $1$ & $+$ & $0$ & $=$ & $1$\\
        $1$ & $+$ & $1$ & $=$ & $10$\\
      \end{tabular}

      \hrule

      \begin{tabular}{rrrrrrrrl}
            & $0$ & $0$ && $0$ & $1$ & $0$ & $1$ & $=5$\\
        ADD & $0$ & $0$ && $0$ & $1$ & $0$ & $1$ & $=5$\\\cmidrule{1-8}
        \only<2>{$\mbox{C}=0$ & $0$ & $0$ && $1$ & $0$ & $1$ & $0$ & $=10$}
      \end{tabular}
    \end{center}

    \note[item]<2->{the hardware has a special bit known as the \emph{carry}
      bit, denoted by C, which stores a 1 if the result of the addition was a
      carry, and 0 otherwise.}
  %}}}1
  \end{frame}

  \begin{frame}{signed addition}
  %{{{1
    \renewcommand{\arraystretch}{2}
    \begin{center}
      \begin{tabular}{rrrrrrrrl}
            & $0$ & $0$ && $0$ & $1$ & $0$ & $1$ & $=+5$\\
        ADD & $1$ & $0$ && $0$ & $1$ & $0$ & $1$ & $=-5$\\\cmidrule{1-8}
        \only<2>{$\mbox{C}=0$ & $1$ & $0$ && $1$ & $0$ & $1$ & $0$ & $=-10$}
      \end{tabular}
    \end{center}

    \note[item]<2->{what is the problem
      \begin{itemize}
        \item in this case, we had two additional symbols, $+$ and $-$, and we
          were making some assumptions about their behavior.
        \item For example, we know that $5 \mbox{ADD} 5=10$, but what does $+
          \mbox{ADD} -$ equal? --- we have no rule for that in our definition of
          decimal.
        \item we are trying to apply the addition algorithm, when we should be
          applying a different algorithm, called \emph{subtraction}
        \item can we choose a different representation that we can directly use
          the addition algorithm with?
      \end{itemize}
    }
  %}}}1
  \end{frame}

  \begin{frame}{one's complement}
  %{{{1
    \renewcommand{\arraystretch}{2}
    \begin{center}
      \begin{tabular}{rrrrrrrr}
              NOT & $0$ & $0$ && $0$ & $1$ & $0$ & $1$\\\hline
        \only<2->{& $1$ & $1$ && $1$ & $0$ & $1$ & $0$}
      \end{tabular}

      \only<3->{\hrule}

      \begin{tabular}{rrrrrrrr}
        \only<3->{    & $0$ & $0$ && $0$ & $1$ & $0$ & $1$\\}
        \only<3->{ADD & $1$ & $1$ && $1$ & $0$ & $1$ & $0$\\\hline}
        \only<4->{$\mbox{C}=0$ & $1$ & $1$ && $1$ & $1$ & $1$ & $1$\\}
        \only<5->{ADD & $0$ & $0$ && $0$ & $0$ & $0$ & $1$\\\hline}
        \only<6->{$\mbox{C}=1$ & $0$ & $0$ && $0$ & $0$ & $0$ & $0$}
      \end{tabular}
    \end{center}

    \note[item]<2->{one's complement is known as logical not}
    \note[item]<4->{adding the one's complement will always result in all 1s}
    \note[item]<6->{so two's complement is $\mbox{NOT}+1$}
  %}}}1
  \end{frame}

  \begin{frame}{cpu bits}
  %{{{1
    \renewcommand{\arraystretch}{2}
    \begin{center}
      \begin{tabular}{r|l|l}
        & \multicolumn{1}{c|}{\textbf{0}} & \multicolumn{1}{c}{\textbf{1}}\\
        \hline
        \textbf{N} & otherwise & result is negative\\
        \textbf{Z} & otherwise & result is all zeros\\
        \textbf{V} & otherwise & signed integer overflow occurred\\
        \textbf{C} & otherwise & unsigned integer overflow occurred\\
      \end{tabular}
    \end{center}

    \note[item]{C is set to carry out of leftmost bit}
    \note[item]{V is set to  detects an overflow by comparing the carry into the
      leftmost bit with the C bit. If they are different, an overflow has
      occurred, and V gets 1. If they are the same, V gets 0.}
  %}}}1
  \end{frame}

  \begin{frame}{register transfer language}
  %{{{1
    \begin{center}
      \begin{tabular}{l|c|l}
        operation & RTL symbol\\
        \hline
        AND                  & $\wedge$\\
        OR                   & $\vee$\\
        XOR                  & $\oplus$\\
        NOT                  & $\neg$\\
        Implies              & $\rightarrow$\\
        Transfer             & $\leftarrow$\\
        Bit index            & $\langle~\rangle$\\
        Informal description & \{~\}\\
        Sequential separator & ;\\
        Concurrent separator & ,\\
      \end{tabular}

      \vspace{\baselineskip}

      \only<2>{$\mbox{c}\leftarrow\mbox{a}\oplus\mbox{b};\mbox{N}\leftarrow\mbox{c}<0,\mbox{Z}\leftarrow\mbox{c}=0$}

      \note[item]<2>{XOR}
    \end{center}
  %}}}1
  \end{frame}

  \begin{frame}{another example}
  %{{{1
    \begin{center}
      \begin{tabular}{r@{}lrrrrrrr}
        \only<1->{&     & 0 & 0 && 0 & 1 & 0 & 1\\}
        \only<1->{& ~ADD & 1 & 1 && 1 & 0 & 1 & 1\\\hline}
        \only<1->{$\mbox{N}\leftarrow$ & ~0 & 0 & 0 && 0 & 0 & 0 & 0\\}
        \only<1->{$\mbox{Z}\leftarrow$ & ~1\\}
        \only<1> {$\mbox{V}\leftarrow$ & ~?\\}
        \only<2> {$\mbox{V}\leftarrow$ & \multicolumn{8}{l}{\hspace{-.25em}$\neg(\mbox{a}\langle0\rangle\oplus\mbox{b}\langle0\rangle)\wedge(\mbox{a}\langle0\rangle\oplus\mbox{N})$}\\}
        \only<1->{$\mbox{C}\leftarrow$ & ~1\\}
      \end{tabular}
    \end{center}
  %}}}1
  \end{frame}

  \begin{frame}{arithmetic shift}
  %{{{1
    \begin{description}
      \item[arithmetic shift left (asl)]\hfill \\
        $\mbox{C}\leftarrow\mbox{r}\langle0\rangle,~\mbox{r}\langle0..4\rangle\leftarrow\langle1..5\rangle,~\mbox{r}\langle5\rangle\leftarrow0;$\\
        $\mbox{N}\leftarrow\mbox{r}<0,~\mbox{Z}\leftarrow\mbox{r}=0,~\mbox{V}\leftarrow\{\mbox{overflow}\}$
      \item[arithmetic shift right (asr)]\hfill\\
        \only<1>{?}
        \only<2>{$\mbox{C}\leftarrow\mbox{r}\langle5\rangle,~\mbox{r}\langle1..5\rangle\leftarrow\langle0..4\rangle;$\\
        $\mbox{Z}\leftarrow\mbox{r}=0$}
    \end{description}

    \note[item]{how will you assign V bit? --- what is the RTL?}
    \note[item]{$\mbox{V}\leftarrow\mbox{C}=\mbox{r}\langle0\rangle$}
    \note[item]<+->{what is RTL for ASR?}
    \note[item]<.->{where are the N and V bits for ASR?}
  %}}}1
  \end{frame}

  \begin{frame}{hexadecimal}
  %{{{1
  %}}}1
  \end{frame}

  \begin{frame}{unicode}
  %{{{1
    Hello world.\\
    ¡Hola!, Grüß Gott, Hyvää päivää, Tere õhtust, Bonġu Cześć!, Dobrý den\\
    {\fontspec{FandolSong} 你好, 早晨, こんにちは}

    \only<2>{\rmfamily https://www.paypal.com\\}
    \only<3>{\rmfamily https://www.p\hspace{-.22em}\raisebox{-.11ex}{\fontsize{9.5pt}{0pt}\selectfont\char"0430}\hspace{-.20em}yp\hspace{-.22em}\raisebox{-.11ex}{\fontsize{9.5pt}{0pt}\selectfont\char"0430}\hspace{-.20em}l.com}

    \note[item]{used to access >127 different characters}
    \note[item]{backwards compatible with ASCII, i.e., code points have same
      value in UTF that they have in ASCII}
    \note[item]{comes in different flavors
      \begin{itemize}
        \item UTF-32 --- easier to understand, requires 32bits to represent each
          of the code points (glyphs), but wasteful if you are mostly storing
          ascii characters
        \item UTF-8 --- variable width code, i.e., some bits in the pattern are
          reserved for storing information about the structure of the pattern
          instead of information about the code point
      \end{itemize}
    }
    \note[item]{also define some emojis}
    \note[item]<2->{can also be used for nefarious things}
  %}}}1
  \end{frame}

  \begin{frame}{floating-point}
  %{{{1
    \begin{description}
      \item[IEEE 754] \hfill\\
        \begin{description}
          \setlength{\itemsep}{0pt}
          \item [single precision] 1.8.23 --- excess 127 / 126
          \item [double precision] 1.11.52 --- excess 1023 / 1022
          \item [special values] \hfill\\
            \begin{tabular}{lll}
              & \textbf{exponent} & \textbf{significand}\\
              \hline
              \textbf{zero} & all zeros & all zeros\\
              \hline
              \textbf{denormalized} & all zeros & non-zero\\
              \hline
              \textbf{inifinity} & all ones & all zeros\\
              \hline
              \textbf{not a number (NaN)} & all ones & non-zero\\
              \hline
            \end{tabular}
        \end{description}

      \item [operations that result in NaN] \hfill\\
        \begin{itemize}
          \item The divisions 0/0 and ±∞/±∞
          \item The multiplications 0×±∞ and ±∞×0
          \item The additions ∞ + (−∞), (−∞) + ∞ and equivalent subtractions
        \end{itemize}
    \end{description}

    \note[item]{mantissa $\rightarrow$ significand}
    \note[item]{more bits in exponent $\rightarrow$ more range}
    \note[item]{more bits in significand $\rightarrow$ more precision}
    \note[item]{how to find the range for floating-point numbers}
    \note[item]<2->{draw attention to Figure 3.38 --- make connection with radix
      sort --- floats can be sorted using radix sort by NOT'ing the values then
      treating them like unsigned integers}
  %}}}1
  \end{frame}

  \appendix

  \begin{frame}[c]
  %{{{1
    \begin{center}\ccbysa\end{center}

    except where otherwise noted, this worked is licensed under
    \href{http://creativecommons.org/licenses/by-sa/4.0/}{creative commons
    attribution-sharealike 4.0 international license}
  %}}}1
  \end{frame}
\end{document}
