\documentclass[10pt,t,usenames,dvipsnames]{beamer}
\usetheme{metropolis}           % Use metropolis theme

\ifnotes
  \hypersetup{final}
  \usepackage{pgfpages}
  \setbeamertemplate{note page}[plain]
  \setbeameroption{show notes on second screen=right}
\fi

\usepackage{appendixnumberbeamer}
\usepackage[scale=3]{ccicons}   % creative commons icons
\usepackage{enumitem}
\usepackage{verbatim}

\title{CSCI 310 --- Computer Organization}
\date{}
\author{Jeremy Iverson}
\institute{College of Saint Benedict \& Saint John's University}
\begin{document}
  \maketitle

  \begin{frame}{logistics}
    \begin{itemize}[itemsep=10pt]
      \item instructor
        \begin{itemize}
          \item Jeremy Iverson (\href{mailto:jiverson002@csbsju.edu}{\nolinkurl{jiverson002@csbsju.edu}})
          \item MAIN 258, (320) 363-5542
          \item office hours: MTThF 12--1pm
        \end{itemize}
      \item textbook
        \begin{itemize}
          \item \emph{Computer Systems}, 5th Edition, Warford
        \end{itemize}
      \item website
        \begin{itemize}
          \item \url{https://csbsju.instructure.com/courses/16318}
        \end{itemize}
    \end{itemize}

    \note{
      \begin{itemize}[label=\textbf{\cdot}]
        \item I prefer to be called Jeremy
        \item Encourage questions right away
        \item Emphasize the importance of the Canvas site for finding
          information about the class
        \item office hours
          \begin{itemize}
            \item Mention outlook calendar \& my home page
              \begin{itemize}
                \item For those unfamiliar with Outlook meetings, then they
                  should schedule another way and we will go over this in
                  meeting
              \end{itemize}
          \end{itemize}
        \item go through Canvas page organization quickly
      \end{itemize}
    }
  \end{frame}

  \begin{frame}{objectives}
    \begin{itemize}[itemsep=10pt]
      \item identify and describe the key components of a computer system and
        how the components interact
      \item explain the layered architecture of computer systems and how each
        layer relates to the others
      \item explain how data and programs are represented inside computers
      \item explain the role of operating systems in managing storage, processes
        and programs
    \end{itemize}
  \end{frame}

  \begin{frame}{evaluation}
    \begin{itemize}[itemsep=10pt]
      \item seven (7) assignments
        \begin{itemize}[label=\cdot]
          \item mix of worked problems and programming
          \item do your own work
          \item use each other as resources only
          \item due dates are strict (no partial credit for late assignments)
        \end{itemize}
      \item three (3) topical exams and one (1) cumulative
        \begin{itemize}[label=\cdot]
          \item closed book / open note
        \end{itemize}
      \item point distribution
        \begin{itemize}
          \item assignments: 32\% total
          \item exams: 14\% each
          \item final: 26\%
        \end{itemize}
    \end{itemize}

    \note{Should be student's own work, may work together but must individual
      solutions\\}
    \note{Want to encourage using each other as resources, but do not want some
      students to rely on other students for answers}
  \end{frame}

  \begin{frame}[standout]{\mbox{need more info\dots}}
    see~\href{https://csbsju.instructure.com/courses/16318/assignments/syllabus}{\nolinkurl{course} \nolinkurl{syllabus}}!
  \end{frame}

  \begin{frame}{a word to the wise}
    \begin{center}
      \Large
      $\mbox{computers}\not\equiv\mbox{magic}$\\
      $\mbox{programming}\not\equiv\mbox{magic}$\\[1\baselineskip]
    \end{center}

    \pause

    \begin{alertblock}{banned phrases}
      \vspace{.5ex}
      ``it worked when I tried it earlier''

      ``I don't know, it just works''

      ``but it gives the right answer''
    \end{alertblock}

    \only<2->{\note{this class is not about your amazing ability to program, it
      is about your ability to reason about how the computer understands what
      you are saying}}
  \end{frame}

  \begin{frame}{a dose of honesty}
    We are not learning new problem-solving abstractions.

    \pause

    We are unpacking the abstractions that we depended on in previous courses
    --- seeing how the expressiveness of our favorite programming languages is
    represented and carried out by a computer.

    We are building a foundation for understanding how future abstractions can
    be understood by a computer.

    \pause

    This process is often \textbf{\color{orange}tedious} and
    \textbf{\color{orange}mechanical}.

    \only<1->{\note{Not directly anyway.\\}}
    \only<3->{\note{A computer is not a \emph{creative} device. In fact, it is
      only superior to humans in its ability to carry out tedious operations at
      an incredible speed.}}
  \end{frame}

  \begin{frame}{a good question}
    \begin{block}{\color{OliveGreen}why bother?}
      \begin{itemize}
        \item better debugger
        \item better programmer
        \item better problem-solver
      \end{itemize}
    \end{block}

    \note{
      most of you will neither write nor maintain compilers nor will you program
      in assembly on a regular basis, so then why bother\dots

      \begin{itemize}[itemsep=10pt]
        \item better debugger --- improve tracing skill and have a better idea
          of how things you write are understood by the machine
        \item better programmer --- debugging is a skill that many
          ``programmers'' lack, but you will have improved this skill therefore,
          you will be a better programmer
        \item better problem-solver --- because of your new understanding of the
          orgranization of a computer system you will better understand its
          capabilities and limitations --- you will also gain experience moving
          between levels of abstraction, thus improving your abstract thinking
      \end{itemize}
    }
  \end{frame}

  \begin{frame}{a look ahead}
    \begin{itemize}[itemsep=10pt]
      \item become proficient in C
      \item internalize the von Neumann architecture
      \item develop a mental model for program execution
      \item get very comfortable with binary
    \end{itemize}

    \note{we will spend approx. one (1) week on C in class; that is not enough
      time to master it the level necessary for this class}
  \end{frame}

  \begin{frame}{a bit of parting advice}
    \begin{itemize}
      \item remember that this is a 300-level course
      \pause
      \item if something is confusing, tell me
    \end{itemize}

    \only<1->{\note{I expect you to do some learning on your own and come to
      class prepared to enhance / clarify / correct the understanding the you
      created. I want to help you explore your understanidng of the material,
      not necessarily help you gain a basic understanding of the material\\}}
    \only<2->{\note{otherwise, I am relying on my own experience to decide which
      material to emphasize and/or clarify\\[1\baselineskip]}}
  \end{frame}

  \begin{frame}{teaching philosophy}
    Your job is to empower those you teach; when you do for them what they
    should be doing for themselves, you create dependency rather than
    empowerment.

    It is easy to give in to the frustration that results from seeing amazing
    possibilities for the people you are teaching, and you want it more for them
    than they want it for themselves.

    Don’t give in to that frustration! \\
    \hfill \footnotesize{--- Based on passage from ``Resisting Happiness'' by
      Matthew Kelly}

    \note{I sincerely believe that all of you can master this material and I
      want more than anything else this semester, to help you do that!}
    \note{However, don't expect me to give you answers. And do get frusterated
      with me when I challenge you to discover the answers for yourself.}
  \end{frame}

  \begin{frame}[standout]
    questions?
  \end{frame}

  \appendix

  \begin{frame}{activity}
    \begin{itemize}
      \item download this slide deck and follow the instructions on the next
        slide
    \end{itemize}

    \note{if there is still time, talk about all of the material for this course
      being hosted on GitHub, and how they can access it}
  \end{frame}

  \ifnotes\else
  \begin{frame}{next slide}
    \begin{itemize}
      \item find out what the reading is for tomorrow
    \end{itemize}
  \end{frame}
  \fi

  \begin{frame}[c]
    \begin{center}\ccbysa\end{center}

    except where otherwise noted, this worked is licensed under
    \href{http://creativecommons.org/licenses/by-sa/4.0/}{creative commons
    attribution-sharealike 4.0 international license}
  \end{frame}
\end{document}
